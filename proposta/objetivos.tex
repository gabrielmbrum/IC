\section{Objetivos}

Neste projeto, a meta é investigar o impacto de técnicas modernas de aumento de dados e arquiteturas híbridas na tarefa de classificação de imagens médicas. Para tanto, pretende-se:

\begin{itemize}
    \item 
    Aplicar transformações complexas com variações paramétricas para gerar versões alternativas das imagens, destacando padrões locais e globais;
    \item 
    Utilizar modelos de GANs como técnica comparativa de aumento de dados, a fim de avaliar sua eficácia relativa em relação às transformações complexas;
    \item 
    Implementar uma arquitetura híbrida de classificação baseada na combinação de CNNs e ViT, explorando sua capacidade de representar múltiplas escalas de informação;
    \item
    Avaliar os resultados obtidos por cada estratégia e seus respectivos desempenhos na arquitetura híbrida, visando compreender as justificativas para tais conclusões e sugerir melhorias;
    \item Identificar os parâmetros mais relevantes das transformações complexas e os limites da abordagem híbrida.
\end{itemize}

Em complemento, esta proposta tem aderência em áreas prioritárias do Ministério da Ciência, Tecnologia, Inovações e Comunicações (estabelecidas na Portaria MCTIC nº 1.122/2020, com texto alterado pela Portaria MCTIC nº 1.329/2020), especificamente com as Áreas de Tecnologias para Qualidade de Vida (Saúde) e Tecnologias Habilitadoras (Inteligência Artificial). No que tange à lista dos Objetivos do Desenvolvimento Sustentável (ODS), esta pesquisa apresenta aderência direta com a ODS 9 (indústria, inovação e infraestrutura).