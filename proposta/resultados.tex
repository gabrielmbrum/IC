\section{Resultados Esperados}

Espera-se que a aplicação das transformações complexas, com variações parametrizadas, contribua significativamente para a geração de imagens sintéticas capazes de enriquecer o conjunto de dados original, promovendo uma melhoria mensurável na acurácia da classificação de imagens médicas. A comparação com a técnica de aumento de dados baseada em GANs deverá evidenciar o potencial das transformações complexas como alternativa viável e eficaz. Adicionalmente, o uso de uma arquitetura híbrida de classificação, combinando CNNs e ViT, deverá apresentar bom desempenho na extração de características locais e globais, visando desfrutar dos benefícios das transformações com variações de parâmetros. Espera-se, ainda, identificar os parâmetros mais influentes nas transformações complexas e levantar possíveis limitações da abordagem híbrida adotada, fornecendo direções para aprimoramentos futuros e contribuindo para o avanço de técnicas de classificação automatizada no campo da saúde.
