\section{Introdução}

Com o avanço da informatização no setor da saúde, a produção de imagens médicas se intensificou significativamente, especialmente as imagens histológicas geradas a partir de biópsias e técnicas de coloração de tecidos. Essas imagens desempenham um papel fundamental na detecção precoce, diagnóstico e monitoramento de diversas doenças, como cânceres e condições inflamatórias. No entanto, sua análise manual é uma tarefa complexa, que demanda alto grau de especialização e está sujeita a variações interobservador. Nesse cenário, métodos computacionais para a classificação automática dessas imagens têm se consolidado como uma alternativa promissora para apoiar a tomada de decisões clínicas, contribuindo para a redução de erros e a otimização de recursos. Essa abordagem ganha ainda mais relevância diante da crescente demanda por diagnósticos rápidos e acessíveis, especialmente em regiões com escassez de profissionais especializados, onde soluções baseadas em inteligência artificial se mostram cada vez mais indispensáveis.

Entre os avanços recentes na área, destacam-se as arquiteturas híbridas que combinam \textit{Convolutional Neural Networks} (CNNs) com \textit{Transformers}, aproveitando diferentes níveis de representação da imagem. Estudos como o de Wu et al. \cite{wu2023ctranscnn} evidenciam a eficácia desses modelos em tarefas de classificação multilabel de imagens médicas, superando abordagens tradicionais como ResNet e \textit{Vision Transformers} (ViT). Apesar desse progresso, ainda há uma lacuna no uso de técnicas de \textit{data augmentation} que explorem plenamente o potencial dessas arquiteturas híbridas. Em especial, observa-se uma escassez de abordagens que considerem transformações complexas como estratégia de geração de dados sintéticos, o que representa uma oportunidade de pesquisa relevante.

As redes complexas têm se mostrado eficazes em tarefas de reconhecimento de padrões e análise de texturas, conforme demonstrado por \cite{ribas2024color}. Nessa abordagem, a imagem é modelada como um grafo complexo onde os pixels são representados como vértices e as conexões são estabelecidas com base em distância euclidiana e diferença de intensidade na escala de cinza. O processo converte medidas topológicas da rede (como grau e força dos vértices) em novos mapas de características, gerando representações que destacam padrões texturais em múltiplas escalas - desde micro até macroestruturas - conforme o raio de conexão varia. Enquanto trabalhos como \cite{ribas2024color} utilizam essas representações como entrada para redes neurais randômicas na extração de descritores para classificação, a mesma técnica pode ser adaptada para gerar variações da imagem original, assim caracterizando uma transformação complexa, mantendo suas propriedades estatísticas fundamentais. Essa capacidade é particularmente relevante em domínios com escassez de dados, como imagens médicas \cite{shorten2019survey}, onde a geração de novas amostras sintéticas pode melhorar o desempenho de modelos de aprendizado profundo.

Este trabalho propõe a integração de transformações complexas a uma arquitetura híbrida composta por CNNs e \textit{Transformers}, combinando as capacidades complementares dessas redes: enquanto as CNNs são eficazes na identificação de padrões locais, os Transformers se destacam na captura de relações contextuais globais por meio do mecanismo de autoatenção \cite{peng2021conformerlocalfeaturescoupling}. Essa combinação é especialmente relevante em cenários que exigem múltiplas escalas de análise, como ocorre em imagens médicas transformadas com variações de raio. Dessa forma, busca-se promover uma aprendizagem mais robusta e sensível às nuances presentes nas imagens clínicas, explorando de maneira sinérgica características locais e contextuais.

Diante disso, esta proposta visa investigar a eficácia do uso de transformações complexas como técnica de aumento de dados em modelos híbridos de classificação de imagens histológicas. Ao empregar essa abordagem em conjunto com arquiteturas baseadas em CNNs e Transformers, objetiva-se extrair representações multiescalares mais informativas. Além disso, será realizada uma comparação sistemática com outra técnica contemporânea de aumento de dados amplamente utilizada: as \textit{Generative Adversarial Networks} (GANs). Com isso, pretende-se conduzir uma análise abrangente da efetividade dessas diferentes estratégias no aprimoramento da classificação médica assistida por inteligência artificial.